
\mychapter{Medium}{Medium}
We chose to make our medium, both rod and surrounding liqid, a changeable parameter to see the influence of the density difference and the liquids viscosity on the process.
First of all, we see that depending on the sign of the density difference, we get either sedimentation, or a collection and ordering of the particles at that top of the liquid. We primarily used iron and lithium as rod material as they demonstrate this change of sign when paired up with water as the surrounding material. Interestingly, while iron has eight times the density of water and lithium is much closer at half of water's density, this does not seem to make a huge difference in the timescales of the process. Though iron seems to order itself slower, so the tendency is there.
As an alternative liquid then water, we choose honey. Our main result ther was that the entire processs slows down as the higher viscosity dampenes movement.
