\mychapter{Introduction}{Introduction}
This project started with the intention to understand and reproduce the topics discussed in ~\cite{SED}. It simulates the ssedimentation of rod particles in a two dimensional domain by mean of a Monte-Carlo simulation. Our main goal was to create a working Monte-Carlo formalism, reproduce a visualisation of the Sedimentation and then change parameters and look at the dynamics.
\mysection{Model}
We simulate the sedimentation of rectangular rods in a quadratic domain $[0,s]×[0,s]$. We first deposit $N$ rods rectangular rods with thickness $d = \hat ds$ and length $l = \hat ls$ for some $\hat d,\hat s∈(0,1)$ using a Random Sequential Adsorption model, either until the wanted number of rods per volume $ρ$ is reached or a maximum number of tries is reached. After this we use a Metropolis Hastings Algorithm to simulate the sedimentation dynamics. For each time step we generate a random movement based on Brownian motion as is done in \cite{BD}. Since our case is slightly different, we have different diffusion coefficients. In our case these are given by
\begin{equation}
  \begin{array}{RLL}
    D_‖ &= \frac{k_BT}{f_\|}\\
    D_\perp &= \frac{k_BT}{f_\perp}\\
    D_r &= \frac{k_BT}{f_r}
  \end{array}
\end{equation}
for the Stokes friction coefficients
\begin{equation}
  \begin{array}{RLL}
    f_\| &= \frac{2πηl}{\ln d + γ_\|}\\
    f_\perp &= \frac{4πηl}{\ln d + γ_\perp}\\
    f_r &= \frac{πηl^3}{3(\ln d+γ_r)}
  \end{array}
\end{equation}
where $η$ is the viscosity of the sourrounding medium and $γ_\| \approx -0.207$, $γ_\perp\approx0.839$ and $γ_r\approx-0.662$ are the end correction coefficients as given in~\cite{SED}.
With the diffusion coefficients we generate for a given time step $δt$ the numbers $δ_* = \sqrt{2D_*δt}$ and using these we generate uniformly distributed random movements $u_*∈[-δ_*,δ_*]$ for a single rod which either move the rod parallel to its length axis ($u_\|$), perpendicular to its length axis ($u_\perp$) or rotate the rod ($u_r$). According to the metropolis Hastings algorithm we then calculate the difference in Energy $ΔU$ between the previous and the trial state and accept the movement with a probability of $\min(1,\exp(\frac{-ΔU}{k_BT}))$. the total energy of our system is
\begin{equation}
  \begin{array}{RLL}
    U = \sum_{i=1}^N V_1(r_i) + \sum_{i,j=1}^N V_2(r_i,r_j) ,
  \end{array}
\end{equation}
for the coordinates of the center of the rods $r_i$, the single rod potentials $V_1(r_i)$ and the inter-rod potentials $V_2(r_i,r_j)$. These are given by
\begin{equation}
  \begin{array}{RLL}
    V_1(r_i) = mgz_i
  \end{array}
\end{equation}
where $z_i$ is the height component of $r_i$, $m = (ρ_r - ρ_m)ld^2$, $g$ is the gravitation and
\begin{equation}
  \begin{array}{RLL}
    V_2(r_i,r_j) =& c_e\lt(\kfrac{\<r_j - r_i,e_{i,\|}\>}{l}^2 + \kfrac{\<r_j - r_i,e_{i,\perp}\>}{d}\rt)^3 \\&+  c_e\lt(\kfrac{\<r_i - r_j,e_{j,\|}\>}{l}^2 + \kfrac{\<r_i - r_j,e_{ij\perp}\>}{d}\rt)^3+ Cθ(r_i,r_j)
  \end{array}
\end{equation}
where $c_e$ is a coefficient calibrating the strength of the ellipsoidal potential, $C$ is a very large number we used $10^{200}$ and $θ$ is a function returning one when the two rods $r_i$ and $r_j$ overlaps and zero otherwise. The first two terms form an ellipsoidal Potential scaling with 6th Power in the ellipsoidal radius.
\mysection{Implementation Details}
\mysubsection{Energy Evaluation}
If we evaluate the energy, especially the inter-rod potentials, as above it is a $O(N^2)$ operation, which is not optimal, because, if we cut off the ellipsoidal potential after some distance, the rod-rod interactions are only local. Moreover we only always change a single rod and therefore the $ΔU$ can also basically be calculated by calculating the energy difference of a single rod. To achieve better performance we substructured our medium into smaller cells, of a size which is such that a rod in a cell, after performing a movement\footnote{Since a movement can also be zero, this includes before the movement}, can only always interact with rods in neighbouring cells which we called the patch. This means that the calculation of the difference in energy for a rod from $r_i^t$ changing its position to $r_i^{t+1}$ is
\begin{equation}
  \begin{array}{RLL}
    ΔU = V_1(r_i^{t+1}) - V_1(r_i^t) + \sum_{j∈P(i)}V_2(r_i^{t+1},r_j) - V_2(r_i^t,r_j)
  \end{array}
\end{equation}
This results in a $O(\#P(i))$ computation cost, which especially for small movements and timesteps is much better than the original.
\mysubsection{Collision Detection}
We differ from~\cite{SED} in a way that we do not treat our rods as being just lines, we include their width in the collision detection. The way we do this is that we look for all intersections of the boundary lines of each rod.
\mysection{Evaluation}


\mysubsection{Order Parameter}
We used the same definition of the order parameter as used in ~\cite{SED} to ensure compareability:
$$S_f= \frac{1}{N} \sum_{i=0}^N 2*cos^2(\phi_i)-1$$
This parameter describes the average orientation of the rods, with the focus being on how close to the horizontal or vertical the particles are oriented. Since $\phi_i$ in our model is the angle between the horizontal and the rod, the order parameter goes to 1 for horizontal particles and towards -1 for vertical particles. It does not distinguish between mirror imaged angles, since the order parameter is symetric both around the vertical and the horizontal orientation point. 

\mysubsection{Sedimentation Column Height}

