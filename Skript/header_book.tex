%\usepackage{/home/tovermodus/Documents/Studium/8S/Mixed/notes-Kanschat/notes/mathsim}

\usepackage{amsmath, mathtools}
\usepackage{libertine}
\usepackage{enumitem}
\usepackage{polyglossia}        %Sprache
\usepackage{amssymb, amsfonts, amsthm}      %ganz viel Mathe
\usepackage{bbm}
\usepackage{enumitem}
\usepackage{tabulary}           %Tabellen
\usepackage{fontspec}           %regelt UTF8 Input
\usepackage{tabu, threeparttable}
\usepackage{booktabs}
\usepackage{tikz}
\usepackage{scalerel}
\usepackage{imakeidx}
\usepackage{bm}
\usepackage{pgfplots}
\usepackage{setspace}
\usepackage[chapter]{algorithm}
\usepackage{minted}
%\usepackage[]{algorithm2e}
\usepackage{breqn}
\usepackage{float}
\usepackage{needspace}
\usepackage{tkz-euclide}
\usepackage{xparse}
\usepackage{xstring}
\usepackage{fancyhdr}
\usepackage[backend=biber, style=numeric, giveninits=true]{biblatex}

\setmainlanguage{german}
\setotherlanguage{english}
\renewcommand*{\newunitpunct}{\addcomma\space}

\renewcommand*{\mkbibnamefamily}[1]{\textsc{#1}}
\pgfplotsset{compat=1.16}
\DeclareFieldFormat*{title}{\mkbibemph{#1}}
\DeclareFieldFormat*{citetitle}{\mkbibemph{#1}}
\DeclareFieldFormat{journaltitle}{#1}

\renewbibmacro*{in:}{%
\ifentrytype{article}
{}
{\printtext{\bibstring{in}\intitlepunct}}}

\newbibmacro*{pubinstorg+location+date}[1]{%
\printlist{#1}%
\newunit
\printlist{location}%
\newunit
\usebibmacro{date}%
\newunit}

\renewbibmacro*{publisher+location+date}{\usebibmacro{pubinstorg+location+date}{publisher}}
\renewbibmacro*{institution+location+date}{\usebibmacro{pubinstorg+location+date}{institution}}
\renewbibmacro*{organization+location+date}{\usebibmacro{pubinstorg+location+date}{organization}}

\DeclareFieldFormat[article]{title}{#1}
\usepackage{scalerel}
\definecolor{mygray}{rgb}{0.1,0.1,.4}
\usepackage[colorlinks = true, linkcolor = mygray, citecolor = mygray, urlcolor = mygray]{hyperref}

%\RequirePackage{cleveref}


\renewcommand{\sectionmark}[1]{\markright{\scshape\hfill
\thesection. #1\hfill}}

\makeatletter
\renewcommand{\@seccntformat}[1]{\csname the#1\endcsname. }
\makeatother

%----- Seiten-Layout:
\setlength{\oddsidemargin}{0.46cm}
\setlength{\evensidemargin}{0.46cm} \setlength{\textwidth}{15cm}
\setlength{\topmargin}{-1cm} \setlength{\headheight}{1cm}
\setlength{\headsep}{1cm} \setlength{\topskip}{0cm}
\setlength{\textheight}{22cm} \setlength{\parindent}{0em}
\setlength{\parskip}{0.6em}
\renewcommand{\bottomfraction}{.7}
\swapnumbers

\addtokomafont{part}{\normalfont\Huge\bfseries\scshape\color{mygray}\hspace*{-.5cm}}
\addtokomafont{partnumber}{\normalfont\Huge\bfseries\scshape\color{mygray}\hspace*{-.5cm}}


% \begingroup \setlength {\dimen@ }{#1}\vskip \z@ \@plus \dimen@ \penalty -100\vskip \z@ \@plus -\dimen@ \vskip \dimen@ \penalty 9999\vskip -\dimen@ \vskip \z@skip \endgroup
\makeatletter

\def\@begintheorem#1#2[#3]{
\setlength {\dimen@ }{30pt}\vskip \z@ \@plus \dimen@ \penalty -100\vskip \z@ \@plus -\dimen@ \vskip \dimen@ \penalty 9999\vskip -\dimen@ \vskip \z@skip 
\deferred@thm@head{\the\thm@headfont \thm@indent
\@ifempty{#1}{\let\thmname\@gobble}{\let\thmname\@iden}%
\@ifempty{#2}{\let\thmnumber\@gobble}{\let\thmnumber\@iden}%
\@ifempty{#3}{\let\thmnote\@gobble}{\let\thmnote\@iden}%
\thm@swap\swappedhead\thmhead{#1}{#2}{#3}%
\the\thm@headpunct
\thmheadnl
%  \setlength {\dimen@ }{80pt}\vskip \z@ \@plus \dimen@ \penalty -100\vskip \z@ \@plus -\dimen@ \vskip \dimen@ \penalty 9999\vskip -\dimen@ \vskip \z@skip 
% possibly a newline.
\hskip\thm@headsep
}%
\ignorespaces}
\makeatother
\newtheoremstyle{defin}% <name>
{15pt\topsep}%      Space above
{\topsep}%      Space below
{\color{black}\itshape}%         Body font
{}%         Indent amount (empty = no indent, \parindent = para indent)
{\color{mygray}\large\bfseries\scshape}% Thm head font
{}%        Punctuation after thm head
%{.5em}%     Space after thm head: " " = normal interword space;
{\newline}%       \newline = linebreak
{}%         Thm head spec (can be left empty, meaning `normal')

\newtheoremstyle{satz}% <name>
{15pt\topsep}%      Space above
{\topsep}%      Space below
{\color{black}\itshape}%         Body font
{}%         Indent amount (empty = no indent, \parindent = para indent)
{\color{mygray}\large\bfseries\scshape}% Thm head font
{}%        Punctuation after thm head
%{.5em}%     Space after thm head: " " = normal interword space;
{\newline}%       \newline = linebreak
{\scshape}%         Thm head spec (can be left empty, meaning `normal')

\newtheoremstyle{bemerkung}% <name>
{15pt\topsep}%      Space above
{\topsep}%      Space below
{\color{black}}%         Body font
{}%         Indent amount (empty = no indent, \parindent = para indent)
{\color{mygray}\large\bfseries\scshape}% Thm head font
{}%        Punctuation after thm head
%{.5em}%     Space after thm head: " " = normal interword space;
{\newline}%       \newline = linebreak
{\scshape}%         Thm head spec (can be left empty, meaning `normal')

\theoremstyle{satz}
\newtheorem{sat}{\iflanguage{german}{Satz}{Theorem}}[chapter]
\newtheorem{lem}[sat]{Lemma}
\newtheorem{kor}[sat]{\iflanguage{german}{Korollar}{Corollary}}
\newtheorem{prop}[sat]{Proposition}
\newtheorem{pro}[sat]{Proposition}
\newtheorem{problem}[sat]{\iflanguage{german}{Aufgabe}{}\iflanguage{english}{Problem}{}}
\newtheorem{solution}[sat]{\iflanguage{german}{Lösung}{}\iflanguage{english}{Solution}{}}
\theoremstyle{bemerkung}
\newtheorem{bem}[sat]{\iflanguage{german}{Bemerkung}{Remark}}
\newtheorem{bsp}[sat]{\iflanguage{german}{Beispiel}{Example}}
\theoremstyle{defin}
\newtheorem{defin}[sat]{Definition}


\newcommand\mysectioncaption{}

%---- Kapitelnummern:
%\renewcommand{\thechapter}{\Roman{chapter}}
%---- Gleichungsnummern:
\numberwithin{equation}{chapter}
\renewcommand{\theequation}{\thechapter.\arabic{equation}}
%---- Abbildungen
\numberwithin{figure}{chapter}
\newcommand{\chapterhead}{}
\newcommand{\tocstyle}[1]{\normalfont\scshape{\textrm{#1}}}
\newcommand{\hestyle}[1]{\color{mygray}\normalfont\scshape{\textrm{#1}}}
%---- Tabellen
\numberwithin{table}{chapter}
\pagestyle{fancy}
\fancyhf{}
\fancyfoot[RO]{\scshape\color{mygray}\pagemark\color{black}}
\fancyfoot[LE]{\scshape\color{mygray}\pagemark\color{black}}
\color{black}
\makeatother
\newenvironment{abstract}
{\begin{center}  \Large\color{mygray}\textsc{Abstract}\color{black}\vspace{0.5cm}\\\large\begin{minipage}{.9\textwidth} \begin{singlespace} }
{\end{singlespace} \end{minipage} \end{center} \vspace{12pt}}
\newenvironment{abstrakt}
{\begin{center}  \Large\color{mygray}\textsc{Zusammenfassung}\color{black}\vspace{0.5cm}\\\large\begin{minipage}{.9\textwidth} \begin{singlespace} }
{\end{singlespace} \end{minipage} \end{center} \vspace{12pt}}
\newcommand{\mychapter}[2]{%\clearpage%\thispagestyle{scrheadings}
\color{mygray}
\refstepcounter{chapter}
\chapter*{\Huge\hestyle{\thechapter. #1}}
\addcontentsline{toc}{chapter}{\LARGE\tocstyle{\thechapter. #1}}
\renewcommand{\chapterhead}{\scshape{\textrm{#2}}}
\color{black}}
\newcommand{\mychapters}[1]{%\clearpage%\thispagestyle{scrheadings}
\color{mygray}
\refstepcounter{chapter}
\chapter*{\Huge\hestyle{#1}}
\addcontentsline{toc}{chapter}{\LARGE\tocstyle{#1}}
\renewcommand{\chapterhead}{#1}
\renewcommand\mysectioncaption{#1}
\fancyhead[CE]{\scshape\color{mygray}\hfill #1 \hfill\color{black}}
\fancyhead[CO]{\scshape\color{mygray} \hfill #1 \hfill\color{black}}
\color{black}}
\renewcommand{\thesection}{\Alph{section}}
\newcommand{\mysection}[1]{%\clearpage
\color{mygray} \refstepcounter{section}
\renewcommand\mysectioncaption{\scshape\thesection. #1}
\section*{\LARGE\hestyle{{\thesection. #1}}}
\addcontentsline{toc}{section}{\Large\tocstyle{{\thesection. #1}}}
\fancyhead[CE]{\scshape\color{mygray}\hfill \thechapter. \chapterhead \hfill\color{black}}
\fancyhead[CO]{\scshape\color{mygray} \hfill \mysectioncaption \hfill\color{black}}
\color{black}}
\newcommand{\mysections}[1]{%\clearpage
\color{mygray} \refstepcounter{section}
\renewcommand\mysectioncaption{#1}
\renewcommand{\chapterhead}{#1}
\section*{\LARGE\hestyle{{#1}}}
\fancyhead[CE]{\scshape\color{mygray}\hfill #1 \hfill\color{black}}
\fancyhead[CO]{\scshape\color{mygray} \hfill #1 \hfill\color{black}}
\color{black}}
\newcommand{\mysubsections}[1]{
\color{mygray} \refstepcounter{subsection}
\subsection*{\Large\hestyle{{\thesubsection \ #1}}}
%\subsection{#1}
\color{black}}
\newcommand{\mysubsection}[1]{
\color{mygray} \refstepcounter{subsection}
\subsection*{\Large\hestyle{{\thesubsection \ #1}}}
%\subsection{#1}
\addcontentsline{toc}{subsection}{\Large\tocstyle{{\thesubsection. #1}}}
\color{black}}

%\renewcommand{\thesubsubsection}{\Alph{subsubsection}}
\newcommand{\mysubsubsection}[1]{
\color{mygray} \refstepcounter{subsubsection}
\subsubsection*{\color{mygray}\large\hestyle{{\thesubsubsection \ #1}}}
%\subsubsection{#1}
\color{black}}

\newcommand{\myparagraph}[1]{
\color{mygray} \refstepcounter{paragraph}
\paragraph*{\hestyle{ #1:}}
\color{black}}
\newenvironment{enum}{\vspace*{-0.7em}
\begin{list}
\theenumi
{\usecounter{enumi}
\setlength{\itemindent}{0.7cm}
\setlength{\labelwidth}{0.5cm}\setlength{\leftmargin}{0cm}
\setlength{\labelsep}{0.2cm}\setlength{\rightmargin}{0cm}
\setlength{\parsep}{0.5ex plus0.2ex minus0.1ex}
\setlength{\itemsep}{0ex plus 0.2ex}}}
{\end{list}\vspace*{-0.7em}}
\renewcommand{\theenumi}{{\normalfont\rmfamily\bfseries\color{mygray}(\roman{enumi})\color{black}}}
\newcommand{\nummer}[1]{\renewcommand{\theenumi}{{\normalfont\rmfamily #1}}\item
\renewcommand{\theenumi}  {{\normalfont\rmfamily (\alph{enumi})}} }
\renewcommand{\labelenumi}{\theenumi}
\renewcommand{\labelenumii}{\alph{enumii})}
\renewcommand{\labelitemi}{\color{mygray}$\bullet$\color{black}}
\renewcommand{\labelitemii}{\color{mygray}$\circ$\color{black}}
\renewcommand{\labelitemiii}{\color{mygray}$\cdot$\color{black}}
\renewcommand{\labelitemiv}{\color{mygray}$\ast$\color{black}}

\makeatletter
\def\l@section{\@dottedtocline{1}{1.5em}{1.5em}}
\def\l@subsection{\@dottedtocline{2}{2.5em}{2em}}
\def\l@subsection{\@dottedtocline{2}{2.5em}{2.5em}}

%Weitere Koma-Skript Einstellungen
\renewcommand*{\pnumfont}{%
\color{mygray}\large\scshape\bfseries}
\setkomafont{partentry}{\Large\scshape\bfseries}
%\setkomafont{partentrynumber}{\Large\scshape\bfseries\color{mygray}}
\setkomafont{chapterentry}{\large\scshape\bfseries}
%\setkomafont{chapterentrynumber}{\large\scshape\bfseries\color{mygray}}
%\setkomafont{sectionentry}{\scshape\bfseries}
%\setkomafont{sectionentrynumber}{\scshape\bfseries\color{mygray}}
\KOMAoption{toc}{graduated,bibliography}
\addtokomafont{caption}{\color{mygray}}
\makeatother


\renewcommand{\proofname}{\scshape \color{mygray}\iflanguage{german}{Beweis}{Proof}\color{black}}
\renewcommand\proof[1][\proofname]{%
\par
\pushQED{\qed}%
\normalfont \topsep 6pt plus 6pt\relax
\trivlist
\item[\hskip\labelsep
\bfseries\color{mygray} #1:\color{black}]\needspace{2\baselineskip}\hfill\newline\ignorespaces
}
\newcommand{\myemph}[1]{\emph{\color{mygray}#1}\index{#1}}
\newcommand{\myemphnone}[1]{\emph{\color{mygray}#1}}
\newcommand{\myemphind}[2]{\emph{\color{mygray}#1}\index{#2}}
\renewcommand{\iflanguage}[3]{\ifthenelse{\equal{\languagename}{#1}}{#2}{#3}}
\newcommand\numeq[1]%
{\stackrel{\scriptscriptstyle(\mkern-1.5mu#1\mkern-1.5mu)}{=}}


\newenvironment{absolutelynopagebreak}
  {\par\nobreak\vfil\penalty0\vfilneg
   \vtop\bgroup}
  {\par\xdef\tpd{\the\prevdepth}\egroup
   \prevdepth=\tpd}

 \bibliography{/home/tovermodus/Documents/Studium/Sonstiges/bibliothek}




\DeclareMathOperator*{\Supp}{supp}
\DeclareMathOperator*{\Lin}{lin}
\DeclareMathOperator*{\Ran}{ran}
\DeclareMathOperator*{\Rank}{\iflanguage{german}{rang}{rank}}
\DeclareMathOperator*{\Ind}{ind}
\DeclareMathOperator*{\Eran}{essran}
\DeclareMathOperator*{\Ker}{ker}
\DeclareMathOperator*{\Dist}{dist}
\DeclareMathOperator*{\Det}{det}
\DeclareMathOperator*{\Arg}{Arg}
\DeclareMathOperator*{\Argmin}{argmin}

\DeclareMathOperator*{\Codim}{codim}
\newcommand{\Tr}{\operatorname{tr}\,}
\DeclareMathOperator*{\Span}{span}
\renewcommand{\Vec}{\operatorname{vec}}


\newcommand{\xn}{(x_n)_{n\in\N}}
\newcommand{\yn}{(y_n)_{n\in\N}}
\newcommand{\un}{(u_n)_{n\in\N}}
\newcommand{\fn}[1]{(#1)_{n\in\N}}
\newcommand{\limx}{\lim_{x\to \infty}}
\newcommand{\limn}{\lim_{n\to \infty}}
\newcommand{\limk}{\lim_{k\to \infty}}
\renewcommand{\mod}{~\operatorname{mod}~}
\NewDocumentCommand{\diri}{O{a} O{n}}{\sum_{#2=1}^{\infty}\frac{#1(#2)}{#2^s}}
\NewDocumentCommand{\sumn}{O{0} O{\infty} O{n}}{\sum_{#3=#1}^{#2}}
\NewDocumentCommand{\sumk}{O{0} O{\infty} O{k}}{\sum_{#3=#1}^{#2}}
\NewDocumentCommand{\sumi}{O{0} O{\infty} O{i}}{\sum_{#3=#1}^{#2}}
\NewDocumentCommand{\summ}{O{0} O{\infty} O{m}}{\sum_{#3=#1}^{#2}}
\NewDocumentCommand{\prodn}{O{0} O{\infty}}{\prod_{n=#1}^{#2}}
\NewDocumentCommand{\prodk}{O{0} O{\infty}}{\prod_{k=#1}^{#2}}
\NewDocumentCommand{\ppot}{O{k} O{r}}{p_1^{#2_1}\cdots p_{#1}^{r_#1}}
\NewDocumentCommand{\xpot}{O{x} O{k} O{r}}{#1_1^{#3_1}\cdots #1_{#2}^{r_#2}}
\newcommand{\prodp}{\prod_{p~prim}}
\newcommand{\eulerp}[1][a]{\prod_{p~prim}\left(1+\sum_{\nu=1}^\infty\frac{#1}{p^{\nu s}}\right)}
\newcommand{\norm}[1]{\lt\|#1\rt\|}
\newcommand{\eb}[1]{\frac{1}{#1}}
\newcommand{\eh}[1]{\frac{#1}{2}}
\newcommand{\keb}[1]{\left(\frac{1}{#1}\right)}
\newcommand{\kfrac}[2]{\left(\frac{#1}{#2}\right)}
\newcommand{\limt}{\lim_{t\to \infty}}
\newcommand{\limh}{\lim_{h\to 0}}
\newcommand{\meng}[2]{\left\{#1\middle|~#2\right\}}
\newcommand{\xnk}{(x_{n_k})_{k\in\N}}
\newcommand{\hatw}{\widehat}
\newcommand{\D}{\mathcal{D}}
\renewcommand{\L}{\mathcal{L}}
\newcommand{\Div}{\operatorname{div}}
\renewcommand{\div}{\Div}
\newcommand{\Grad}{\mathrm{D}}
\newcommand{\I}{\operatorname{Id}}
\newcommand{\Id}{\operatorname{Id}}
\newcommand{\Rot}{\operatorname{\iflanguage{german}{rot}{curl}}}
\newcommand{\Diag}{\operatorname{diag}}
\newcommand{\<}{\left\langle}
\newcommand{\wto}{\rightharpoonup}
\newcommand{\wtos}{\overset{\ast}{\rightharpoonup}}
\newenvironment*{§}{$\displaystyle}{$}
\renewcommand{\>}{\right\rangle}
\newcommand{\R}{\mathbb{R}}
\newcounter{subeqqq}
\renewcommand{\thesubeqqq}{\theequation\alph{subeqqq}}
\newcommand{\thesubequation}{\thesubeqqq}
\makeatletter
\newcommand\slabel[1]{%
\@bsphack
\if@filesw
{\let\thepage\relax
\def\protect{\noexpand\noexpand\noexpand}%
\edef\@tempa{\write\@auxout{\string
\newlabel{#1}{{\thesubeqqq}{\thepage}}}}%
\expandafter}\@tempa
\if@nobreak \ifvmode\nobreak\fi\fi
\fi\@esphack}
\makeatother


\newcolumntype{R}{>{\displaystyle}r}
\newcolumntype{L}{>{\displaystyle}l}
\newcolumntype{C}{>{\displaystyle}c}
\newcolumntype{N}{>{\stepcounter{subeqqq}\quad\mathrm{(\thesubeqqq)}}r}
\newcolumntype{M}{>{\stepcounter{subeqqq}\quad\mathrm{(\thesubeqqq)}}r}
\newenvironment{arraynum}[1]
{
    \setcounter{subeqqq}{0}
  \refstepcounter{equation}
  \begin{equation*}
      \begin{array}{#1 Nc}
        }
        {
      \end{array}
  \end{equation*}
}

\newenvironment{arraynumr}[1]
{
    \setcounter{subeqqq}{0}
  \refstepcounter{equation}
  \begin{equation*}
    %\begin{array}{>{\raggedleft\arraybackslash}p{\linewidth}}
      \begin{array}{#1 Nc}
        }
        {
      \end{array}
    %\end{array}
  \end{equation*}
}

        
        
%\setlist{enumerate}{topsep=0pt}
\newcounter{equinum}
\newcounter{eqlnum}
\newcounter{eqrnum}
\newcommand{\labelequiv}[1]{\def\equivname{#1}\setcounter{equinum}{1}\setcounter{eqrnum}{1}\setcounter{eqlnum}{1}}
\newcommand{\eqitem}{\item \def\equinumber{\alph{equinum}}\label{\equivname\equinumber}\stepcounter{equinum}}

\newcommand{\eqritem}{\item[\ref{\equivname\alph{eqrnum}}$\Rightarrow$\stepcounter{eqrnum}\ifnum\value{eqrnum}=\value{equinum}\setcounter{eqrnum}{1}\fi\ref{\equivname\alph{eqrnum}}]}
\newcommand{\eqlitem}{\item[\ref{\equivname\alph{eqlnum}}\stepcounter{eqlnum}]}
\newcommand{\eqraitem}[2]{\item[\ref{\equivname#1}$\Rightarrow$\ref{\equivname#2}]}


\renewcommand*{\C}{\mathbb{C}}
\newcommand{\Q}{\mathbb{Q}}
\newcommand{\K}{\mathbb{K}}
\newcommand{\N}{\mathbb{N}}
\newcommand{\Z}{\mathbb{Z}}
\renewcommand{\AA}{\mathbb{A}}
\newcommand{\DD}{\mathbb{D}}
\newcommand{\MM}{\mathbb{M}}
\renewcommand{\a}{~\textnormal{\iflanguage{german}{auf}{on} }}
\renewcommand{\i}{~\textnormal{in }}
\newcommand{\ex}{~\textnormal{\iflanguage{german}{existiert}{exists} }}
\newcommand{\fa}{\quad\textnormal{\iflanguage{german}{für alle}{for all} }}
\newcommand{\st}{\quad\textnormal{s.t. }}
\newcommand{\f}{\quad\textnormal{\iflanguage{german}{für}{for} }}
\newcommand{\so}{\quad\textnormal{\iflanguage{german}{sonst}{else} }}
\newcommand{\fü}{\quad\textnormal{\iflanguage{german}{f.ü.}{a.e.} }}
\newcommand{\m}{~\textnormal{\iflanguage{german}{mit}{with} }}
\newcommand{\E}{\mathbbm{1}}
\renewcommand{\O}{\mathcal{O}}
\renewcommand{\Re}{\operatorname{Re}}
\renewcommand{\Im}{\operatorname{Im}}
\renewcommand{\O}{\mathcal{O}}
\newcommand{\RR}{\mathcal{R}}
\newcommand{\A}{\mathcal{A}}
\newcommand{\M}{\mathcal{M}}
\newcommand{\W}{\mathcal{W}}
\newcommand{\Schwarz}{\mathcal{S}}
\newcommand{\bepsilon}{\varepsilon}
\newcommand{\bphi}{\varphi}
\renewcommand{\epsilon}{\varepsilon}
\renewcommand{\arraystretch}{1.3}
\newcommand{\mydots}[1]{\IfBeginWith{#1}{,}{\ldots#1}{\cdots#1}}
%Klammern
\newcommand{\lt}{\left}
\newcommand{\rt}{\right}
\newcommand{\mt}{\middle}
\arraycolsep=2pt
\makeatletter
\RenewDocumentEnvironment{cases}{ O{l>{\quad}l} O{1.2} }
 {\let\@ifnextchar\new@ifnextchar
  \left\lbrace
  \def\arraystretch{#2}%
  \array{#1}}
 {\endarray\right.}
\makeatother
\newcommand{\mat}[2][6pt]{\arraycolsep=#1\begin{pmatrix}#2\end{pmatrix}}
\newcommand{\smat}[1]{\left(\begin{smallmatrix}#1\end{smallmatrix}\right)}
\catcode`ω=13
\catcode`ℵ=13
\catcode`ε=13
\catcode`ζ=13
\catcode`τ=13
\catcode`θ=13
\catcode`ι=13
\catcode`ρ=13
\catcode`π=13
\catcode`α=13
\catcode`σ=13
\catcode`δ=13
\catcode`φ=13
\catcode`γ=13
\catcode`η=13
\catcode`∂=13
\catcode`ξ=13
\catcode`κ=13
\catcode`λ=13
\catcode`υ=13
\catcode`χ=13
\catcode`ψ=13
\catcode`β=13
\catcode`ν=13
\catcode`μ=13
\catcode`Ω=13
\catcode`Θ=13
\catcode`Π=13
\catcode`Σ=13
\catcode`Δ=13
\catcode`Φ=13
\catcode`Γ=13
\catcode`Ξ=13
\catcode`Λ=13
\catcode`Ψ=13
\catcode`⇒=13
\catcode`⇔=13
\catcode`∈=13
\catcode`⊂=13
\catcode`∪=13
\catcode`∩=13
\catcode`∃=13
\catcode`∀=13
\catcode`⇐=13
\catcode`∋=13
\catcode`⊃=13
\catcode`×=13
\catcode`⋅=13
\catcode`∇=13
\catcode`‾=13
\catcode`∞=13
\catcode`⊗=13
\catcode`…=13
\def∞{\infty}
\def∇{\Grad}
\defΨ{\Psi}
\def‾{\overline}
\defω{\omega}
\defℵ{\aleph}
\defε{\epsilon}
\defζ{\zeta}
\defτ{\tau }
\defθ{\theta}
\defι{\iota}
\defρ{\rho}
\defπ{\pi}
\defα{\alpha}
\defσ{\sigma}
\defδ{\delta}
\defφ{\varphi}
\defγ{\gamma}
\defη{\eta}
\def∂{\partial}
\defξ{\xi}
\defκ{\kappa}
\defλ{\lambda}
\defυ{\upsilon}
\defχ{\chi}
\defψ{\psi}
\defβ{\beta}
\defν{\nu}
\defμ{\mu}
\defΩ{\Omega} 
\defΘ{\Theta}
\defΠ{\Pi}
\defΣ{\Sigma}
\defΔ{\Delta}
\defΦ{\Phi}
\defΓ{\Gamma}
\defΞ{\Xi}
\defΛ{\Lambda}
\def⇒{\Rightarrow}
\def⇔{\Leftrightarrow}
\def∈{\in}
\def⊂{\subset}
\def∪{\cup}
\def∩{\cap}
\def∃{\exists}
\def∀{\forall}
\def⇐{\Leftarrow}
\def∋{\ni}
\def⊃{\supset}
\def×{\times}
\def⋅{\cdot}
\def⊗{\otimes}
\def…{\mydots}
\catcode`↑=13
\catcode`←=13
\catcode`↓=13
\catcode`→=13
\catcode`↑=13
\catcode`←=13
\catcode`↓=13
\catcode`→=13

\catcode`〈=13
\catcode`〉=13
\catcode`†=13
\def〉{\rangle}
\def〈{\langle}
\def†{\dagger}
\def↑{\uparrow}
\def←{\leftarrow}
\def↓{}
\def→{}
\makeatletter \renewcommand\d{\mathrm{d}}
\makeatother
\makeatletter
\DeclareRobustCommand\ddt[1]{\frac{\d #1}{\d t}}
\makeatother
\makeatletter
\DeclareRobustCommand\dex[1]{\frac{∂ #1}{∂x}}
\makeatother
\makeatletter
\DeclareRobustCommand\det[1]{\frac{∂ #1}{∂t}}
\makeatother
\makeatletter
\def\active@math@prime{\hspace*{1em}\kern-1em^\bgroup\prim@s}
{\catcode`\'=\active \global\let'\active@math@prime}
\def\prim@s{%
  \prime\futurelet\@let@token\pr@m@s}
\def\pr@m@s{%
  \ifx'\@let@token
    \expandafter\pr@@@s
  \else
    \ifx^\@let@token
      \expandafter\expandafter\expandafter\pr@@@t
    \else
      \egroup
    \fi
  \fi}
\def\pr@@@s#1{\prim@s}
\def\pr@@@t#1#2{#2\egroup}
\DeclareRobustCommand\widecheck[1]{{\mathpalette\@widecheck{#1}}}
\def\@widecheck#1#2{%
    \setbox\z@\hbox{\m@th$#1#2$}%
    \setbox\tw@\hbox{\m@th$#1%
       \widehat{%
          \vrule\@width\z@\@height\ht\z@
          \vrule\@height\z@\@width\wd\z@}$}%
    \dp\tw@-\ht\z@
    \@tempdima\ht\z@ \advance\@tempdima2\ht\tw@ \divide\@tempdima\thr@@
    \setbox\tw@\hbox{%
       \raise\@tempdima\hbox{\scalebox{1}[-1]{\lower\@tempdima\box
\tw@}}}%
{\ooalign{\box\tw@ \cr \box\z@}}}
\def\tagform@#1{\maketag@@@{(\ignorespaces#1\unskip)}}
  \makeatother


